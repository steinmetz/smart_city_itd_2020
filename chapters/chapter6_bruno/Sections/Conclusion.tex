
\section{Conclusion}

\subsection{Results}\label{6sec:Results}
%   show what is working
% \\ which requirements are not implemented?
% \\ what could you have done in a different way?

The implementation works as it should.
The user can add add or remove as many NPCs or manual drivers as they want.
All requirements are fulfilled in the final version,
with the caveat mentioned above in Section \ref{6sec:manCar}.
When a manual driver crashes into something, 
the messages are successfully send to the MQTT server.
The filter also works well, 
though it could be argued that the strict four 
second threshold is too simplistic.

\noindent 
There is at the time of writing a problem with the communication with the server,
where the accident reports are not correctly registered.
But this seems to be a relatively simple matter 
of getting the right formatting,
either on the server or the client end.
But overall, the goal of creating a smart city 
that monitors the car safety of its citizens has been achieved.

\subsection{Future work}\label{6sec:future Work}

Currently a lot of information is send to the server that is completely ignored.
Values like the heart rate, the intensity of the crash
or the object that the car crashed into.
Further functionality could be added to the server and
the ambulance projects, to take these into account.
This could help make better decisions in cases 
where there is a high demand for emergency responses,
but not enough capacity a the facilities.

To the crash detection program, a feature to detect if the crash partner is also 
a manual car could be added.
This could make it so that only one emergency message is send to the hospital,
not two for the same crash.
\\
\newline
The biggest improvement to the system  would be to and the crash monitoring to NPCs.
This could make this project a true simulation of a smart city,
when the system doesn't require a user to crash Manual Cars into walls.
To push the total simulation thought even further,
the hospitals could also be implemented in CARLA,
so the ambulances actually respond to the message and drive to the location.